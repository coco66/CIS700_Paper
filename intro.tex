In recent years, numerous neural network based methods have been introduced for automatic dialog generation. \cite{Sutskever} presented a sequence to sequence learning model with deep neural networks (SEQ2SEQ) consisting of two multi-layered Long Short-Term Memory (LSTM) and showed that it outperformed a standard SMT-based system on an English to French translation task. \cite{Vinyals} applied this sequence to sequence framework to conversational modeling and their model predicted the next sentence given the previous sentence. Unlike other models that had been used widely, this model achieved better performance requiring much fewer hand-crafted rules. \\
However, since the SEQ2SEQ model was originally proposed for machine translation tasks, it does not capture previous conversations when it is applied to a conversation task. Being able to address previously mentioned topics or information is essential in conversation. In order to solve this issue, \cite{Serban} extended the hierarchical recurrent encoder-decoder (HRED), proposed by \cite{Sordoni}, to the dialog domain. Although they used only triple utterances for implementation, they were able to show their proposed model outperformed both $n$-gram based models and baseline neural network models. Another issue of the SEQ2SEQ model is that it tends to be short-sighted since the model does not consider its future outcomes. Addressing this issue, \cite{Li} proposed a novel approach for a dialog generation task combining a policy gradient optimization method and the SEQ2SEQ model (DRL-SEQ2SEQ). The policy gradient optimization method is one of policy-based methods in Reinforcement Learning (RL), and it updates its policy parameters in order to maximize a learner's expected discounted future total reward. Thus, we can expect the model would generate an outcome more likely to continue the current conversation. \\
While the HRED model does not consider its future outcomes, the DRL-SEQ2SEQ model cannot address its conversation history other than a few immediate previous utterances. In this project, we studied the HRED model for dialog generation in open domain proposed by \cite{Serban} and DRL-SEQ2SEQ. We initially aimed to replace SEQ2SEQ model in DRL-SEQ2SEQ with HRED in order to utilize the advantage of each model. However, due to the time constraint, we were not able to combine both models but tried implementing them on OpenSubtitle Dataset. 